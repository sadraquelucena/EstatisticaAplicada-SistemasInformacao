\documentclass[11pt,a4paper]{report}
\usepackage{PlanoEnsinoDECAT}
\usepackage[brazil]{babel}
\usepackage{color}
\usepackage[normalem]{ulem}
\usepackage{enumitem}
\usepackage{float}

\usepackage[table]{xcolor}
\usepackage{booktabs} % para linhas na tabela colorida

\usepackage[
top    = 2.50cm,
bottom = 2.50cm,
left   = 2.60cm,
right  = 2.50cm]{geometry}

\usepackage{caption}
\captionsetup[table]{name=Quadro}

\usepackage{palatino} % fonte
\usepackage{setspace} % espaçamento


\codigo{ESTAT0011}                      % codigo da disciplina
\disciplina{ESTATÍSTICA APLICADA}  % nome da disciplina
\docente{Sadraque E. F. Lucena}         % nome do(a) docente
\cargahoraria{60 horas}                 % carga horaria da disciplina
\creditos{4 cr\'editos}                 % quatidade de creditos da disciplina
\horario{Segundas - 20h45 \`as 22h15} % horario 1
\horarioo{Quartas - 20h45 \`as 22h15}   % horario 2
\periodo{2025-1}                        % periodo letivo
\qtdaval{3}                             % quantidade de avaliacoes Ex.: 2

\begin{document}

\cabecalho

\begin{itemize}[label={},itemindent=-2em,leftmargin=2em]
    \item {\bf Ementa:} Introdução. Regras elementares de probabilidade. Distribuição binominal, Poisson e normal. População e amostras. Testes de bondade de ajustamento. Uso de transformações. Distribuições de certas estatísticas amostrais. Noções de testes e hipóteses. Noções de delineamento experimental. Experimentos com um e dois fatores. Regressão e correlação.
    
    \item {\bf Objetivos:} Proporcionar experiências de aprendizagem que permitam ao estudante familiarizar-se com conhecimentos estatísticos fundamentais para a análise, interpretação e solução de problemas cotidianos e aplicados.
    
    \item {\bf Metodologia:} A disciplina será desenvolvida por meio de aulas teóricas expositivas, com uso de recursos visuais, resolução de exercícios em sala e realização de atividades extraclasse. O cronograma de aulas encontra-se no Quadro 1.
    
    \item {\bf Habilidades e Competências:} Ao final da disciplina, o estudante conhecerá os principais conceitos de estatística descritiva, terá noções básicas de probabilidade e inferência, e compreenderá os fundamentos de correlação e regressão linear simples. Estará, assim, apto a aplicar essas técnicas na resolução de problemas cotidianos e em contextos práticos.
    
    \item {\bf Avaliação:} Serão realizadas três avaliações ao longo do semestre, podendo consistir em provas, trabalhos ou outras atividades definidas pelo docente.
\end{itemize}

\noindent {\bf Conteúdo:}
\begin{enumerate}
	\singlespacing
	\item Estatística Descritiva
	\begin{enumerate}[label*=\arabic*.]
		\item As fases do trabalho estatístico;
		\item Classificação dos dados;
		\item Séries estatísticas e sua representação tabular e gráfica;
		\item Distribuição de frequências simples e em classe e sua representação gráfica;
		\item Medidas de Posição: Média, Mediana e Moda;
		\item Medidas de Dispersão: Variância, desvio-padrão, coeficiente de variação.
	\end{enumerate}
	\vspace{.3cm}
	\item Probabilidade
	\begin{enumerate}[label*=\arabic*.]
		\item Noções de conjuntos;
		\item Conceitos de probabilidade: Experimento aleatório, espaço amostral e eventos;
		\item Eventos independentes;
		\item Probabilidade condicional e teorema de Bayes.
		\item Esperança matemática, variância e desvio-padrão;
		\item Distribuições discretas: Bernoulli, Binomial, Geométrica e Poisson;
		\item Distribuições contínuas: Uniforme, Exponencial e Normal.
	\end{enumerate}
	\item Inferência
	\begin{enumerate}[label*=\arabic*.]
		\item Noções de Amostragem;
		\item Conceitos iniciais de inferência;
		\item Distribuição da média e da proporção;
		\item Intervalo de confiança para média e proporção;
		\item Testes de hipóteses para média e proporção.
	\end{enumerate}
	\item Correlação e Regressão Linear.
	\begin{enumerate}[label*=\arabic*.]
		\item Tipos de correlação;
		\item Correlação Linear de Pearson;
		\item Regressão Linear Simples;
		\item Intervalo de confiança para média e proporção;
		\item Estimativa de parâmetros e interpretação.
	\end{enumerate}
\end{enumerate}

\noindent {\bf Bibliografia:}\\

\noindent {\it Básica:}
\begin{itemize}
     \item MAGALHÃES, Marcos Nascimento; LIMA, Antônio Carlos Pedroso de. Noções de probabilidade e estatística. 7. ed. atual. São Paulo, SP: EDUSP, 2010.
     
     \item MORETTIN, Luiz Gonzaga. Estatística básica: probabilidade e inferência. São Paulo: Pearson, 2010.
     
     \item MEYER, P. L.. Probabilidade. Aplicações à Estatística. Livros Técn. Científicos, 1972.
     
     \item ROSS, Sheldon M. Probabilidade: um curso moderno com aplicações. 8. ed. Porto Alegre, RS: Bookman, 2010.
     
     \item VIEIRA, Sonia. Estatística básica. São Paulo: Cengage Learning, 2012.
     
\end{itemize}

\noindent {\it Complementar:}
\begin{itemize}
    \item ROSS, Sheldon M. Introduction to probability models. 8th. ed. United States of America: Academic Press, 2003.
    
    \item MAGALHÃES, Marcos Nascimento; Probabilidade e Variável Aleatória. 3. ed. São Paulo, SP: EDUSP, 2011.
\end{itemize}

\begin{table}[H]
	\caption{Cronograma de aulas de Estatística Computacional para o período 2025-1.}
	\rowcolors{1}{gray!20}{white}
	\renewcommand{\arraystretch}{1.15}
	\begin{tabular}{|c|c|c|p{10cm}|}
		\toprule
		\textbf{Data}	&	\textbf{Dia da semana}	&	\textbf{Aula}	&	\textbf{Assunto Previsto}	\\
		\midrule
		12/05/2025 & Segunda & 1 & Apresentação da disciplina \\
		14/05/2025 & Quarta & 2 & Medidas de localização \\
		19/05/2025 & Segunda & 3 & Medidas de variabilidade \\
		21/05/2025 & Quarta & 4 & Gráficos \\
		26/05/2025 & Segunda & 5 & Atividade \\
		28/05/2025 & Quarta & 6 & Atividade \\
		02/06/2025 & Segunda & 7 & Atividade \\
		04/06/2025 & Quarta & 8 & \textbf{Avaliação 1} \\
		09/06/2025 & Segunda & 9 & Probabilidade. Eventos e suas probabilidades. Regras de Probabilidade.\\
		11/06/2025 & Quarta & 10 & Análise Combinatória (parte 1) \\
		16/06/2025 & Segunda & 11 & Análise Combinatória (parte 2) \\
		18/06/2025 & Quarta & 12 & Probabilidade Condicional e Independência. \\
		23/06/2025 & Segunda & - & \textcolor{red}{Véspera de Sâo João (recesso acadêmico)} \\
		25/06/2025 & Quarta & 13 & Exercícios. \\
		30/06/2025 & Segunda & 14 & Variáveis Aleatórias Discretas. Valor Esperado e Variância. \\
		02/07/2025 & Quarta & 15 & Distribuição de Bernoulli e Binomial. \\
		07/07/2025 & Segunda & 16 & Distribuição Geométrica, Binomal Negativa e Poisson. \\
		09/07/2025 & Quarta & 17 & Variáveis Aleatórias Contínuas.\\
		14/07/2025 & Segunda & 18 & Distribuição Uniforme e Normal.\\
		16/07/2025 & Quarta & 19 & Exercícios.\\
		21/07/2025 & Segunda & 20 & Exercícios.\\
		23/07/2025 & Quarta & 21 & Exercícios.\\
		28/07/2025 & Segunda & 22 & \textbf{Avaliação 2}\\
		30/07/2025 & Quarta & 23 & Intervalos de Confiança para a Média.\\
		04/08/2025 & Segunda & 24 & Intervalos de Confiança para a Proporção.\\
		06/08/2025 & Quarta & 25 & Teste de Hipóteses para a Média de uma Amostra.\\
		11/08/2025 & Segunda & 26 & Teste de Hipóteses para a Proporção.\\
		13/08/2025 & Quarta & 27 & Teste de Hipóteses para a Diferença de Médias.\\
		18/08/2025 & Segunda & 28 & Exercícios.\\
		20/08/2025 & Quarta & 29 & Exercícios.\\
		25/08/2025 & Segunda & 30 & \textbf{Avaliação 3} \\
		01/09/2025 & Segunda & - & \textbf{Avaliação Repositiva} \\
		\bottomrule
	\end{tabular}
\end{table}

\noindent {\bf Hora-trabalho:} Ao final de cada aula serão indicados exercícios para os alunos resolverem como forma de fixação do conteúdo exposto em sala de aula.

\end{document}